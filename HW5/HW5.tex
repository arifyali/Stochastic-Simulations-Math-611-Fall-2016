\documentclass{article}\usepackage[]{graphicx}\usepackage[]{color}
%% maxwidth is the original width if it is less than linewidth
%% otherwise use linewidth (to make sure the graphics do not exceed the margin)
\makeatletter
\def\maxwidth{ %
  \ifdim\Gin@nat@width>\linewidth
    \linewidth
  \else
    \Gin@nat@width
  \fi
}
\makeatother

\definecolor{fgcolor}{rgb}{0.345, 0.345, 0.345}
\newcommand{\hlnum}[1]{\textcolor[rgb]{0.686,0.059,0.569}{#1}}%
\newcommand{\hlstr}[1]{\textcolor[rgb]{0.192,0.494,0.8}{#1}}%
\newcommand{\hlcom}[1]{\textcolor[rgb]{0.678,0.584,0.686}{\textit{#1}}}%
\newcommand{\hlopt}[1]{\textcolor[rgb]{0,0,0}{#1}}%
\newcommand{\hlstd}[1]{\textcolor[rgb]{0.345,0.345,0.345}{#1}}%
\newcommand{\hlkwa}[1]{\textcolor[rgb]{0.161,0.373,0.58}{\textbf{#1}}}%
\newcommand{\hlkwb}[1]{\textcolor[rgb]{0.69,0.353,0.396}{#1}}%
\newcommand{\hlkwc}[1]{\textcolor[rgb]{0.333,0.667,0.333}{#1}}%
\newcommand{\hlkwd}[1]{\textcolor[rgb]{0.737,0.353,0.396}{\textbf{#1}}}%
\let\hlipl\hlkwb

\usepackage{framed}
\makeatletter
\newenvironment{kframe}{%
 \def\at@end@of@kframe{}%
 \ifinner\ifhmode%
  \def\at@end@of@kframe{\end{minipage}}%
  \begin{minipage}{\columnwidth}%
 \fi\fi%
 \def\FrameCommand##1{\hskip\@totalleftmargin \hskip-\fboxsep
 \colorbox{shadecolor}{##1}\hskip-\fboxsep
     % There is no \\@totalrightmargin, so:
     \hskip-\linewidth \hskip-\@totalleftmargin \hskip\columnwidth}%
 \MakeFramed {\advance\hsize-\width
   \@totalleftmargin\z@ \linewidth\hsize
   \@setminipage}}%
 {\par\unskip\endMakeFramed%
 \at@end@of@kframe}
\makeatother

\definecolor{shadecolor}{rgb}{.97, .97, .97}
\definecolor{messagecolor}{rgb}{0, 0, 0}
\definecolor{warningcolor}{rgb}{1, 0, 1}
\definecolor{errorcolor}{rgb}{1, 0, 0}
\newenvironment{knitrout}{}{} % an empty environment to be redefined in TeX

\usepackage{alltt}
\usepackage{amscd, amssymb, amsmath, verbatim, setspace}
\usepackage[left=1.0in, right=1.0in, top=1.0in, bottom=1.0in]{geometry}
\usepackage{mathrsfs}
\usepackage{listings}


\IfFileExists{upquote.sty}{\usepackage{upquote}}{}
\begin{document}
\begin{flushright}
Arif Ali\\
Math 611 Stochastic Simulation\\
Oct 06, 2016\\
\end{flushright}

\begin{center}
\LARGE\textbf{Homework 5}
  \end{center}
\section*{Exercise 1}
$1,2,3$ are all aperiodical because $gcd(1)=gcd(2)=gcd(3)=1$
\section*{Exercise 2}
States $1,2$ are periodic with $gcd(1)=gcd(2)=2$ so the period for both stages is also 2.
\section*{Exercise 3}

\begin{equation} 
\begin{split}
f(x|\theta) = \frac{1}{\theta+\frac{1}{2}-(\theta-\frac{1}{2})}= 1
\end{split}
\end{equation}
From $f(x|\theta)$, we know that $L(\theta|x) = \prod_{i=1}^{n} 1^{n} =1$. This means that there is no unique value for $\theta$ 

However $X_{(1)} > \theta -\frac{1}{2}$ and $X_{(n)} < \theta+ \frac{1}{2}$

From Pitman 319, $f_{min}(x)=n(1-F(x))^{n-1}f(x)$ and $f_{max}(x)=n(F(x))^{n-1}f(x)$
\begin{equation}
\begin{split}
X_{(1)} = min(X_1...X_n) \sim n(1/2+\theta-x)^{n-1} \\
X_{(n)} = max(X_1...X_n) \sim n(1/2-\theta+x)^{n-1}
\end{split}
\end{equation}
The MLE of $\theta$ should occur when $X_{(1)}=X_{(n)}$

\begin{equation}
\begin{split}
n(1/2+\theta-x_{(1)})^{n-1} = n(1/2-\theta+x_{(n)})^{n-1} \implies \\
1/2+\theta-x_{(1)} = 1/2-\theta+x_{(n)} \implies \\
\theta-x_{(1)} = -\theta+x_{(n)} \implies \\
\theta = \frac{x_{(1)} + x_{(n)}}{2}
\end{split}
\end{equation}
$\hat{\theta} \sim \frac{X_{(1)} + X_{(n)}}{2}$
\section*{Exercise 4}
\subsection*{Part A}
Let the prior be $\theta \sim unif(0,1)$ and $X|\theta \sim Bernoulli(\theta)$
\begin{equation}
\begin{split}
p(\theta|X) \propto \prod_{i=1}^{n} p(X_{i}|\theta) * p(\theta) = \\
\prod_{i=1}^{n} \theta^{x_i}(1-\theta)^{1-x_i} = \\
\theta^{\sum_{i=1}^{n}x_i}(1-\theta)^{\sum_{i=1}^{n}1-x_i} = \\
\theta^{\sum_{i=1}^{n}x_i}(1-\theta)^{n-\sum_{i=1}^{n}x_i}
\end{split}
\end{equation}
Let $y = \sum_{i=1}^{n}x_i$ so,
\begin{equation}
\begin{split}
\theta^{y}(1-\theta)^{n-y} = \theta^{(y+1)-1}*(1-\theta)^{(n-y+1)-1}\propto beta(y+1, n-y+1)
\end{split}
\end{equation}
$\therefore$ the Bayes Estimate is $\frac{y+1}{y+1+n-y+1}=\frac{y+1}{n+2}$
\subsection*{Part B}
\begin{equation}
\begin{split}
L(\theta) =\prod_{i=1}^{n} \theta^{x_i}(1-\theta)^{1-x_i} \implies \\
Log(L(\theta)) = \sum_{i=1}^{n} log(\theta^{x_i}(1-\theta)^{1-x_i}) = \\
\sum_{i=1}^{n} log(\theta^{x_i}) + \sum_{i=1}^{n} log((1-\theta)^{1-x_i}) = \\
\sum_{i=1}^{n} x_i log(\theta) + \sum_{i=1}^{n} {1-x_i} * log(1-\theta) = \\
n\bar{x}*log(\theta) + (n-n\bar{x})* log(1-\theta)
\end{split}
\end{equation}
\begin{equation}
\begin{split}
\delta/d\theta(log(L(\theta))) = \frac{n\bar{x}}{\theta} - \frac{n-n\bar{x}}{1-\theta} = 0 \implies \\
\frac{n\bar{x}}{\theta} = \frac{n-n\bar{x}}{1-\theta} \implies \\ 
n\bar{x}*(1-\theta) = (n-n\bar{x})(\theta) \implies \\
n\bar{x} = n\theta
\end{split}
\end{equation}
$\therefore \hat{\theta}=\bar{x} = \frac{y}{n}$

Since the prior distribution is uniform, $E(\theta) = \frac{1}{2}$
\begin{equation}
\begin{split}
\frac{1}{2}(1-\omega)+\frac{y}{n}(\omega) = \frac{y+1}{n+2} \implies \\
\omega = \frac{n}{n+2} \implies \\
1-\omega =\frac{2}{n+2}
\end{split}
\end{equation}

\begin{equation}
\frac{1}{2}(\frac{2}{n+2})+\frac{y}{n}(\frac{n}{n+2}) = \frac{y+1}{n+2}
\end{equation}


\section*{Exercise 5}
Let $f(x;\theta)=e^{-(x-\theta)}$, $Y_{1}=min(X_{i})$, and $T = T(X_{1},...,X{n})$

\begin{equation}
F(x) = \int_{\theta}^{x} e^{-(x-\theta)}dx = 1-e^{\theta-x}
\end{equation}
From Pitman 319, $f_{min}(x)=n(1-F(x))^{n-1}f(x)$

\begin{equation}
\begin{split}
f_{min}(x) = n(1-(1-e^{\theta-x}))^{n-1}e^{-(x-\theta)} = \\
n(e^{\theta-x})^{n-1}e^{-(x-\theta)} = \\
n*(e^{\theta-x})^{n}
\end{split}
\end{equation}

\begin{equation}
\begin{split}
P(X_{1},....,X_{n}|T=t)= \frac{e^{n\theta-\sum x_{i}}}{n*(e^{\theta-t})^{n}} = \\
\frac{1}{n}e^{n\theta-\sum x_{i}-(\theta-t)n} = \\
\frac{1}{n}e^{-\sum x_{i}-(-t)n}
\end{split}
\end{equation}
Since the conditional density and range do not depend on $\theta$, $Y_{1}=min(X_{i})$ is a sufficent Statistic

\end{document}
