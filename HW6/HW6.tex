\documentclass{article}\usepackage[]{graphicx}\usepackage[]{color}
%% maxwidth is the original width if it is less than linewidth
%% otherwise use linewidth (to make sure the graphics do not exceed the margin)
\makeatletter
\def\maxwidth{ %
  \ifdim\Gin@nat@width>\linewidth
    \linewidth
  \else
    \Gin@nat@width
  \fi
}
\makeatother

\definecolor{fgcolor}{rgb}{0.345, 0.345, 0.345}
\newcommand{\hlnum}[1]{\textcolor[rgb]{0.686,0.059,0.569}{#1}}%
\newcommand{\hlstr}[1]{\textcolor[rgb]{0.192,0.494,0.8}{#1}}%
\newcommand{\hlcom}[1]{\textcolor[rgb]{0.678,0.584,0.686}{\textit{#1}}}%
\newcommand{\hlopt}[1]{\textcolor[rgb]{0,0,0}{#1}}%
\newcommand{\hlstd}[1]{\textcolor[rgb]{0.345,0.345,0.345}{#1}}%
\newcommand{\hlkwa}[1]{\textcolor[rgb]{0.161,0.373,0.58}{\textbf{#1}}}%
\newcommand{\hlkwb}[1]{\textcolor[rgb]{0.69,0.353,0.396}{#1}}%
\newcommand{\hlkwc}[1]{\textcolor[rgb]{0.333,0.667,0.333}{#1}}%
\newcommand{\hlkwd}[1]{\textcolor[rgb]{0.737,0.353,0.396}{\textbf{#1}}}%
\let\hlipl\hlkwb

\usepackage{framed}
\makeatletter
\newenvironment{kframe}{%
 \def\at@end@of@kframe{}%
 \ifinner\ifhmode%
  \def\at@end@of@kframe{\end{minipage}}%
  \begin{minipage}{\columnwidth}%
 \fi\fi%
 \def\FrameCommand##1{\hskip\@totalleftmargin \hskip-\fboxsep
 \colorbox{shadecolor}{##1}\hskip-\fboxsep
     % There is no \\@totalrightmargin, so:
     \hskip-\linewidth \hskip-\@totalleftmargin \hskip\columnwidth}%
 \MakeFramed {\advance\hsize-\width
   \@totalleftmargin\z@ \linewidth\hsize
   \@setminipage}}%
 {\par\unskip\endMakeFramed%
 \at@end@of@kframe}
\makeatother

\definecolor{shadecolor}{rgb}{.97, .97, .97}
\definecolor{messagecolor}{rgb}{0, 0, 0}
\definecolor{warningcolor}{rgb}{1, 0, 1}
\definecolor{errorcolor}{rgb}{1, 0, 0}
\newenvironment{knitrout}{}{} % an empty environment to be redefined in TeX

\usepackage{alltt}
\usepackage{amscd, amssymb, amsmath, verbatim, setspace}
\usepackage[left=1.0in, right=1.0in, top=1.0in, bottom=1.0in]{geometry}
\usepackage{mathrsfs}
\usepackage{listings}


\IfFileExists{upquote.sty}{\usepackage{upquote}}{}
\begin{document}
\begin{flushright}
Arif Ali\\
Math 611 Stochastic Simulation\\
Oct 13, 2016\\
\end{flushright}

\begin{center}
\LARGE\textbf{Homework 6}
  \end{center}
\section*{Exercise 1}
\begin{knitrout}
\definecolor{shadecolor}{rgb}{0.969, 0.969, 0.969}\color{fgcolor}\begin{kframe}
\begin{alltt}
\hlstd{x} \hlkwb{=} \hlkwd{c}\hlstd{(}\hlnum{169.14353}\hlstd{,} \hlnum{135.73850}\hlstd{,} \hlnum{102.46566}\hlstd{,}  \hlnum{80.91151}\hlstd{,} \hlnum{148.45425}\hlstd{,}
      \hlnum{144.68948}\hlstd{,} \hlnum{106.56257}\hlstd{,} \hlnum{104.83559}\hlstd{,}\hlnum{94.81216}\hlstd{,} \hlnum{109.47048}\hlstd{,}
      \hlnum{95.94150}\hlstd{,} \hlnum{123.84673}\hlstd{,}  \hlnum{87.18401}\hlstd{,} \hlnum{104.73420}\hlstd{,} \hlnum{111.94364}\hlstd{,}
      \hlnum{119.69467}\hlstd{,} \hlnum{151.77627}\hlstd{,}  \hlnum{81.80692}\hlstd{,} \hlnum{116.58660}\hlstd{,}  \hlnum{98.28933}\hlstd{)}

\hlstd{mu_1} \hlkwb{<-} \hlkwd{c}\hlstd{(}\hlnum{100}\hlstd{)}
\hlstd{mu_2} \hlkwb{<-} \hlkwd{c}\hlstd{(}\hlnum{120}\hlstd{)}
\hlcom{## as well as the latent variable parameters}
\hlcom{#tau_1 <- c(1-0.7) # 1-W}
\hlstd{W} \hlkwb{<-} \hlkwd{c}\hlstd{(}\hlnum{0.7}\hlstd{)} \hlcom{# W}
\hlkwa{for}\hlstd{( i} \hlkwa{in} \hlnum{1}\hlopt{:}\hlnum{12}\hlstd{) \{}
  \hlcom{## Given the observed data, as well as the distribution parameters,}
  \hlcom{## what are the latent variables?}
  \hlstd{T_1} \hlkwb{<-} \hlstd{(}\hlnum{1}\hlopt{-}\hlstd{W[i])} \hlopt{*} \hlkwd{dnorm}\hlstd{(x, mu_1[i],} \hlkwc{sd} \hlstd{=} \hlkwd{sqrt}\hlstd{(}\hlnum{20}\hlstd{))}
  \hlstd{T_2} \hlkwb{<-} \hlstd{W[i]} \hlopt{*} \hlkwd{dnorm}\hlstd{(x, mu_2[i],} \hlkwc{sd} \hlstd{=} \hlkwd{sqrt}\hlstd{(}\hlnum{25}\hlstd{))}
  \hlstd{P_1} \hlkwb{<-} \hlstd{T_1} \hlopt{/} \hlstd{(T_1} \hlopt{+} \hlstd{T_2)}
  \hlstd{P_2} \hlkwb{<-} \hlstd{T_2} \hlopt{/} \hlstd{(T_1} \hlopt{+} \hlstd{T_2)} \hlcom{## note: P_2 = 1 - P_1}
  \hlcom{#tau_1[i+1] <- mean(P_1)}
  \hlstd{W[i}\hlopt{+}\hlnum{1}\hlstd{]} \hlkwb{<-} \hlkwd{mean}\hlstd{(P_1)}
  \hlcom{## Given the observed data, as well as the latent variables,}
  \hlcom{## what are the population parameters}
  \hlstd{mu_1[i}\hlopt{+}\hlnum{1}\hlstd{]} \hlkwb{<-} \hlkwd{sum}\hlstd{( P_1} \hlopt{*} \hlstd{x)} \hlopt{/} \hlkwd{sum}\hlstd{(P_1)}
  \hlstd{mu_2[i}\hlopt{+}\hlnum{1}\hlstd{]} \hlkwb{<-} \hlkwd{sum}\hlstd{( P_2} \hlopt{*} \hlstd{x)} \hlopt{/} \hlkwd{sum}\hlstd{(P_2)}
\hlstd{\}}
\hlkwd{data.frame}\hlstd{(mu_1, mu_2, W)[}\hlopt{-}\hlnum{1}\hlstd{,]}
\end{alltt}
\begin{verbatim}
##         mu_1     mu_2         W
## 2   96.04482 133.4018 0.5074671
## 3   98.20225 138.5979 0.5979239
## 4   99.30204 140.9658 0.6365587
## 5   99.85074 142.2978 0.6561922
## 6  100.33812 143.4546 0.6728337
## 7  100.80388 144.7254 0.6894349
## 8  101.02890 145.4078 0.6977058
## 9  101.07789 145.5545 0.6994714
## 10 101.08642 145.5785 0.6997678
## 11 101.08782 145.5824 0.6998159
## 12 101.08804 145.5830 0.6998237
## 13 101.08808 145.5831 0.6998249
\end{verbatim}
\end{kframe}
\end{knitrout}
\section*{Exercise 2}
\subsection*{Part A}
Since $X\sim N(0,1) \implies X^2 \sim Chi-squared(1)$ then $Z_{1}^{2} \sim Chi-squared(1)$.

By modifying the derivation from http://www.math.uah.edu/stat/special/ChiSquare.html:
please note $f(x)$ and $F(x)$ refer to the PDF and CDF of $N(0,1)$
\begin{equation}
\begin{split}
p(X=x | X\sim \theta_{1} Z_{1}^{2}) \implies \\
p(-\sqrt{x/\theta_1}<Z_1<\sqrt{x/\theta_1}) = 2*F(\sqrt{x/\theta_{1}}) - 1
\end{split}
\end{equation}

\begin{equation}
\begin{split}
d/\delta x(2*F(\sqrt{x/\theta_{1}})-1)= \\
2*1/2*1/\sqrt{(}\theta_{1})*x^{1/2-1}*f(\sqrt{x/\theta_{1}})=\\
1/\sqrt{\theta_{1}}*x^{1/2-1}*\frac{1}{\sqrt{2\pi}}e^{(\sqrt{x/\theta_{1}})^{2}/2}=\\
\frac{1}{\sqrt{\pi}}*\frac{1}{\sqrt{2\theta_{1}}}*x^{1/2-1}*e^{\frac{1}{2\theta_{1}}x} =\\
\frac{1}{\sqrt{\pi}}*(\frac{1}{2\theta_{1}})^{1/2}*x^{1/2-1}*e^{\frac{1}{2\theta_{1}}x} \sim Gamma(\frac{1}{2},\frac{1}{2\theta_{1}})
\end{split}
\end{equation}
\subsection*{Part B}
From Part A, we know that $\theta_{1}*Z_{1}^{2} \sim Gamma(\frac{1}{2},\frac{1}{2\theta_{1}})$. Using this same logic, $\theta_{2}*Z_{2}^{2} \sim Gamma(\frac{1}{2},\frac{1}{2\theta_{2}})$, so by using Method of momments on one of the theta parameters, it should mirror the results of the other theta parameter.
\begin{equation}
\begin{split}
E(\theta_{1}*Z_{1}^{2}{}^{1})=E(\theta_{1}*Z_{1}^{2})=\frac{\frac{1}{2}}{\frac{1}{2\theta_{1}}}=\\
\theta_{1}E
(\theta_{1}*Z_{1}^{2})= \\
Var(\theta_{1}*Z_{1}^{2})+(E(\theta_{1}*Z_{1}^{2}))^{2}=\\
2(\theta_{1})^{2}+\theta_{1}^{2}=3\theta_{1}^{2}
\end{split}
\end{equation}
\subsection*{Part C}
\begin{knitrout}
\definecolor{shadecolor}{rgb}{0.969, 0.969, 0.969}\color{fgcolor}\begin{kframe}
\begin{alltt}
\hlstd{theta_1} \hlkwb{=} \hlnum{0.3}
\hlstd{theta_2} \hlkwb{=} \hlnum{1}\hlopt{-}\hlstd{theta_1}
\hlstd{W} \hlkwb{=} \hlkwd{c}\hlstd{()}
\hlkwa{for}\hlstd{(i} \hlkwa{in} \hlnum{1}\hlopt{:}\hlnum{1e3}\hlstd{)\{}
  \hlkwa{if}\hlstd{(}\hlkwd{runif}\hlstd{(}\hlnum{1}\hlstd{)}\hlopt{<} \hlstd{theta_1)\{}
    \hlstd{W[i]} \hlkwb{=} \hlstd{theta_1}\hlopt{*}\hlkwd{rchisq}\hlstd{(}\hlnum{1}\hlstd{,}\hlnum{1}\hlstd{)}
  \hlstd{\}} \hlkwa{else} \hlstd{\{}
    \hlstd{W[i]} \hlkwb{=} \hlstd{theta_2}\hlopt{*}\hlkwd{rchisq}\hlstd{(}\hlnum{1}\hlstd{,}\hlnum{1}\hlstd{)}
  \hlstd{\}}
\hlstd{\}}

\hlkwa{for}\hlstd{( i} \hlkwa{in} \hlnum{1}\hlopt{:}\hlnum{12}\hlstd{) \{}
  \hlcom{## Given the observed data, as well as the distribution parameters,}
  \hlcom{## what are the latent variables?}
  \hlstd{T_1} \hlkwb{<-} \hlkwd{dgamma}\hlstd{(W,}\hlnum{0.5}\hlstd{,} \hlnum{1}\hlopt{/}\hlstd{(}\hlnum{2}\hlopt{*}\hlstd{theta_1))}
  \hlstd{T_2} \hlkwb{<-} \hlkwd{dgamma}\hlstd{(W,}\hlnum{0.5}\hlstd{,} \hlnum{1}\hlopt{/}\hlstd{(}\hlnum{2}\hlopt{*}\hlstd{theta_2))}
  \hlstd{P_1} \hlkwb{<-} \hlstd{T_1} \hlopt{/} \hlstd{(T_1} \hlopt{+} \hlstd{T_2)}
  \hlstd{P_2} \hlkwb{<-} \hlstd{T_2} \hlopt{/} \hlstd{(T_1} \hlopt{+} \hlstd{T_2)}

  \hlstd{theta_1[i}\hlopt{+}\hlnum{1}\hlstd{]} \hlkwb{<-}  \hlkwd{sum}\hlstd{( P_1} \hlopt{*} \hlstd{W)} \hlopt{/} \hlkwd{sum}\hlstd{(P_1)}
  \hlstd{theta_2[i}\hlopt{+}\hlnum{1}\hlstd{]} \hlkwb{<-} \hlkwd{sum}\hlstd{( P_2} \hlopt{*} \hlstd{W)} \hlopt{/} \hlkwd{sum}\hlstd{(P_2)}

\hlstd{\}}
\hlkwd{data.frame}\hlstd{(theta_1, theta_2)[}\hlopt{-}\hlnum{1}\hlstd{,]}
\end{alltt}
\begin{verbatim}
##      theta_1   theta_2
## 2  0.3300647 0.8869361
## 3  0.3317461 0.8940444
## 4  0.3327202 0.8963451
## 5  0.3332192 0.8975735
## 6  0.3335825 0.8982493
## 7  0.3339054 0.8986130
## 8  0.3340001 0.8990231
## 9  0.3343020 0.8991573
## 10 0.3342301 0.8994933
## 11 0.3344372 0.8995990
## 12 0.3347586 0.8994143
## 13 0.3346795 0.8996898
\end{verbatim}
\end{kframe}
\end{knitrout}


\section*{Exercise 3}
\subsection*{Part A}
\begin{equation}
\left(\begin{array}{cccc}
0.5 & 0.5 & 0 & 0\\
0.05 & 0.5 & 0.45 & 0\\
0 & 0.1 & 0.5 & 0.4\\
0 & 0 & 0.3 & 0.7
\end{array}\right)
\end{equation}

\subsection*{Part B}
Let $\pi(0) = c$
\begin{equation}
\begin{split}
\pi(1) = \pi(0)*\frac{p_{0}}{q_{1}}=\pi(0)*\frac{0.5}{0.05} = 10c \\
\pi(2) = \pi(1)*\frac{.45}{0.1}=45c \\
\pi(3) = \pi(2)*\frac{.4}{0.3}=60c
\end{split}
\end{equation}

\begin{equation}
\begin{split}
\pi(0)+\pi(1)+\pi(2)+\pi(3)=1= \\
c+10c+45c+60c = 116c \implies \\
c=\frac{1}{116}
\end{split}
\end{equation}
So:
\begin{equation}
\begin{split}
\pi(0)+\pi(1)+\pi(2)+\pi(3)=1= \\
c+10c+45c+60c = 116c \implies \\
c=\frac{1}{116}
\end{split}
\end{equation}

\begin{equation}
\begin{split}
\pi(0) = \frac{1}{116}\\
\pi(1) = \frac{10}{116}\\
\pi(2) = \frac{45}{116}\\
\pi(3) = \frac{60}{116}
\end{split}
\end{equation}
\section*{Exercise 4}
From class, we know that:

\begin{equation}
\left(\begin{array}{cccc}
0 & 3/3 & 0 & 0\\
1/3 & 0 & 2/3 & 0\\
0 & 2/3 & 0 & 1/3\\
0 & 0 & 3/3 & 0
\end{array}\right)
\end{equation}
And that the stationary distribution follows a binomial distribution:
\begin{equation}
\pi(i) = \left(\begin{array}{c}
n\\
i
\end{array}\right)
 = (\frac{1}{2})^{i}*(\frac{1}{2})^{n-i}= \pi(i) = \left(\begin{array}{c}
n\\
i
\end{array}\right)
(\frac{1}{2})^{n}
\end{equation}

\begin{equation}
\begin{split}
p(i)*p(i,i+1)=\left(\begin{array}{c}
n\\
i
\end{array}\right)
(\frac{1}{2})^{n}*\frac{n-i}{n} = \\
\frac{n!}{i!*(n-i)!} (\frac{1}{2})^{n}*\frac{n-i}{n} = \\
\frac{n*(n-1)!}{i!*(n-i)(n-i-1)!} (\frac{1}{2})^{n}*\frac{n-i}{n} = \\
\frac{(n-1)!}{i!*(n-i-1)!} (\frac{1}{2})^{n} = \\
\frac{n!}{(i+1)!*(n-i-1)!} (\frac{1}{2})^{n}*\frac{i+1}{n} = \\
\pi(i+1)p(i+1,i)
\end{split}
\end{equation}
\end{document}
